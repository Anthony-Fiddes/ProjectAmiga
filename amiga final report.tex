\documentclass[10pt,american english]{article}
\usepackage[a4paper,bindingoffset=0.2in,%
            left=1in,right=1in,top=1in,bottom=1in,%
            footskip=.25in]{geometry}
\usepackage{blindtext}
\title{Technical Documentation: Project Amiga}
\author{Alex Welsh, Anna Mola, Anthony Fiddes (PO), Caroline Yoon\\ Julia Levine, Mario Becerra Aleman, Rahul Nair (SM)}
\date{November 30, 2020}
\begin{document}
\maketitle
% Start with introduction, then add the sections with explanations for each one
\section*{Introduction}
\textbf{This technical documentation aims to explicate the process, features, and technologies that were used during the development of our semester-long project for the\textit{ CS 370: Software Engineering Practicum} course.}
\newline \newline 
\textit{Amiga} was designed to be a mobile application that presents individuals with a friendly, user-centric design that enables logging and mood tracking to visualize historical progress of mental health. The intention is for users to gain an understanding of how real-world events impact their mood. In turn, this empowers users to identify and control stressors and stress-reducers, and to manage their mental health in a more proactive way. Such ``journalling'' can also be shared with medical providers and loved ones to show trends in mental health over time. 
\newline \newline 
Each section will explain, in more technical detail, each titular feature, and will walk through the design, development, and execution/integration of said feature into the application as a whole.
\begin{enumerate}
\item Agile, Technical Beginnings, and How We Work
\item Front- and Back-End: Design and Development
\item Logging: Keeping Track of Moods
\item Stats Page
\item Gamification: Friends, Streaks, and Achievements
\item Resources
\item Profile Details
\item Miscellaneous Technical Details
\end{enumerate}

\section*{Agile, Technical Beginnings, and How We Work}

\subsection*{Agile}
At the beginning of the term, after the course's Agile exam and team formation, the team met to align on a project idea. After having unanimously agreed on the mental health-focused \textit{Amiga}, the team was unable to identify a formalized ``business'' entity in the scope of our course. Therefore, the team itself became the business stakeholders, and the Product Owner and Scrum Master worked in tandem with ''business'' to decide on an overall ``project vision.''
\newline
\newline
Crucial to the team's determination and subsequent execution of project scope was our use of our Agile board. Hosted on Microsoft's Azure DevOps ("ADO") platform, the board and linked product backlog served as a transparent way for the team to visualize current and future work items, align on project scope, and understand who was working on what, thereby facilitating productivity and load management. It became the responsibility of the Scrum Master to establish and lead team ceremonies, such as Backlog Refinement, Team Retrospectives, and Daily Stand-Ups, to ensure the team was in constant alignment, dealt with minimal dependencies, and moved forward together.
\newline
\newline
As is common at the outset of real-world software development projects, the team held an initial meeting to populate work items for the entirety of the term in something akin to a \textit{Program Increment} refinement session. This initial session of backlog population served as a foundation upon which the development timeline of the project was created. The team determined the features necessary for the final deliverables, prioritized said features, estimated total development times for each, allocated each to one of ten sprints, and discussed the medium of delivery for the final project. What came of this session was a robust product backlog, which informed a rough development timeline.
\newline
\newline
One of the key Agile guidelines the team agreed upon was to keep the board up-to-date with updates in PBI status, scope, and definition of done. The Scrum Master took special care to enforce these rules while challenging the team to maintain a constant sprint velocity of one feature/sprint. This meant that one large work item (usually a collection of three or more PBIs or 15+ effort points) was expected to be completed each sprint. This ensured that the team remained motivated and completed agreed-upon work to an agreed-upon standard.

\subsection*{Techstack}
The first three sprints were dedicated to the completion of "Spike" (research) product backlog items (PBIs), which laid the foundation for important decisions that were made further along during the project timeline. One major SPIKE item was the determination of the techstack used; after the team aligned that \textit{Amiga} should be a mobile app, there was doubt as to which technology should be learned and leveraged to best execute the project vision. The leading candidates were Flutter, Xamarin, and React Native. During the first three sprints, the team split into Design and Development sub-teams; Design was tasked with using Adobe Xd and open-source images to create a mock-up design of what the app would look like on both Android and iOS, while the Development team split each member to a techstack and had a reasoned discussion to determine a techstack for the long-term execution of the project. More detail on this process will come in the ``Front- and Back-end Design'' section.

\subsection*{Working Together}
Every week saw the team complete slides in preparation for a Sprint Review session to the larger course audience, and the team would take that time to also complete code reviews. Using GitHub for Source Code Management, the team submitted Pull Requests (PR) with the intent of committing feature branch changes into the mainstream \textbf{master} branch. Team members would congregate to read through the PR contents together, and then execute the code on their local machines, testing for errors and bugs on simulated and real-world iPhone and Android devices. Fixes would be implemented as needed, and upon approval, the code would be merged into \textbf{master}, ensuring stable, shippable, and buildable code at all times. 

\section*{Front- and Back-end Design/Development}

\subsection*{Design}
As previously referenced, the initial split of the team into Design and Development sub-teams led to the creation of mock-ups of app structure and user workflow regardless of final techstack. Tools used were \textbf{Adobe Xd}, \textbf{coolors.co}, and several \textbf{open-source libraries} to create vivid designs with functional buttons, headers, menus, and modals for interaction. In particular, the ability of Xd to share these functional mock-ups as detached, mini-web apps for use by the Development team in their creation of the final deliverable was especially valuable. A key component of mapping out user workflows was to imagine how users would transition from one feature to the next; for example, a user may open the app, log their day's mood, and transition to the \textbf{Stats} page to view a day-over-day progression of the logged moods. Such workflow modelling and feature development provided vital blueprints for the Development team to follow as the app matured. The Design team transitioned into the Development team by the end of Sprint 4, with all of the team working on Development in some capacity by Sprint 5.

\subsection*{Development}
Development started with the other sub-team that came out of the initial division of labor at the outset of the project. The first task of the Development team was to determine which techstack to use for the duration of the project. Initially, a lot of momentum around ``newer'' technologies, such as Flutter, motivated discussions, but the team decided that the final techstack should...
 \begin{itemize}
\item be cross-platform.
\item require a short-ish learning curve.
\item execute quickly and easily on the user's device.
\item have robust libraries for integration into back-end techstack.
\item have lots of documentation.
\end{itemize}

After Sprint 3, the determination was made to move to React Native. It represented the best ``learning opportunity'' for the team, as many expressed an interest in learning JavaScript, HTML, and CSS. The other pre-requisites fulfilled, the next big step was to make determinations on how to build out the back-end. 
\newline
\newline
The team considered establishing a MERN (MongoDB, ExpressJS, React, Node) techstack, but meant that several key components would need to be instantiated. This was quickly was struck down due to the limited time available for the project's execution. As such, the team sought out an out-of-the-box, back-end as a service (BaaS) offering, and the free \textit{Spark} tier of \textbf{Google Firebase} provided an excellent solution that seamlessly integrated with React Native.
\newline
\newline
Thanks to the NoSQL nature of Firebase's \textbf{Cloud Firestore} tool, the team realized significant time-savings from avoiding the instantiation of database credentials and access, routes, and key relations that the MERN stack would have necessitated. Instead, \textbf{Firestore} established data hierarchically, starting with sub-collections at the root and creating documents as branches. Each document nested customizable fields underneath it, and this structure enabled robust, but compute-efficient sub-collections to be created for users (which contained \textbf{Friends}, \textbf{Achievements}, and \textbf{Streaks}) and log (moods, journal entries, and mood slider values). With a front-end development framework complemented by a back-end to store and. manage the resulting user data, development of our core features could mature during Sprints 7, 8, and 9.

\section*{Logging: Keep Track of Moods}
The main technical function of \textit{Amiga} involved the writing and saving of logs. Users should be able to open the app, login, land on the home screen, write a journal entry, visualize historical log metrics, and close the app when done. This simple workflow was split into multiple steps that took several sprints to complete. Their technical details are explained further as follows:
\begin{itemize}
\item login: handled by \textbf{Firebase} authentication, the team designed a login page and ``create user'' sign up form for new users and password resets. Importantly, the usernames were all moderated by email address, and functionality for ``usernames'' or ``handles'' was deemed unnecessary.
\item navigation: handled by React Native navigation.
\item journal entry: this was the most complex part of the home screen and arguably, of the app as a whole. The front-end pieces are visualized by React Native components and modals in \textit{LogItem} and \textit{LogModal}, and the back-end storage of these logs reads/writes them in the aforementioned \textbf{Firestore}. Users wrote their entry and were able to select specific ``mood words'' to describe their entry. Here again, users slide a 0-100 slider to quantify their emotions on a numerical scale for easy comparison. 
\item \textbf{Stats}: visualizing mood over time was a key feature, enabled by plotting the scale percentile values against the given date.
\end{itemize}
With the workflow of the user visualized, it was important to note that all these entries are visible. This means that users can show their progress to a licensed mental health professional, a loved one, or most importantly, themselves for personal self-reflection. By helping democratize this information, the user can feel encouraged to find connections between how they feel and how far they have come.

\section*{Visualization in Stats}
\textbf{Stats} is a core function of \textit{Amiga} that enables the user to view progress of mood percentile over time. The graph generated by \textbf{Stats} is a smoothed plot that visualizes mood percentile over time. View windows show day, week, and month iterations when enough data exists to populate those views. A calendar below the graph displays a calendar that displayed each logged date in color, creating a colorful graphic that can also be used to show which days had logs and how the user felt on each of those days. To promote continued use of the app, the \textbf{Stats} page also includes small buttons for \textbf{Streaks} and \textbf{Achievements}.

\section*{Gamification: Friends, Streaks, and Achievements}
Gamification features in mobile apps have made significant changes in the way users interact with each other and with the contents of the app itself. For example, leaderboards in Fitbit and Apple Health have ``gamified'' performance, encouraging use of the app as well as affecting desired behaviors - in this case, diet and exercise - onto the app's userbase. Similar gamification strategies have not yet been implemented for mental health apps as of yet, but \textit{Amiga} adds light gamification by including achievements - statistics that can be viewed by a user's friends. 
\newline
\newline
\textbf{Streaks} are a very simple method of encouraging continued use. Each day logged increases the streak count by 1, and each subsequent day thereafter continues to increment that streak value. The user's longest-ever streak is stored in \textbf{Firebase} as a part of the user profile, and is shared with friends. Furthermore, progressively longer streaks that exceed Fibonacci numbers (1, 2, 3, 5, 8, 13, etc.) will unlock achievements. The total length of the longest streak will be visible to friends. 
\newline
\newline
The team struggled with an important ethical consideration: how would \textit{Amiga} enable friends - sharing personal information - while maintaining the integrity of the users' very personal information; mental health logs are understandably very private, but the team still believed gamification would be an important method of user retention. Therefore, the published level of gamification with simple visualization of Streaks and Furthermore, the team made the determination that if the user would like to share more information, they can do so by sharing their mobile devices as needed.

\section*{Resources}

The \textbf{Resources} feature aims to direct users to nearby landmarks that may improve mental health. At present, \textit{Amiga} allows its users to search between a park, library, cafe, or local therapist. It works by locating the user using in-built GPS location via the user's device. Once location is accurately triangulated, the Google Maps-rendered map is displayed with a blue dot representing the user's location. Selecting a button, like "cafe", will make an API call to the \textit{Google Maps API}, searching locally around that location for a cafe. Similar searches can be run for the other options and can easily direct the user to the selected location.

\section*{Profile Details}
The team leveraged open-source tools to create Avatars for each user to select. The idea was to enable the user to select a cartoon depiction of themselves to use in the app while also maintaining a visible image for other users to connect to them when utilizing the Friends feature of the app.
\newline
\newline
Additionally, the app stores email address as personal information upon the creation of a new account. This is to enable the back-end authentication structure via \textbf{Firebase} using email/password authentication.
\newline
\newline
A final detail baked into the profile is a small, simple to-do list visible on the home screen. Useful for managing tasks and their statuses, the to-do list stores tasks as objects and allows users to cross them off as tasks are completed.

\section*{Testing and Feedback}
The team used close friends, loved ones, classmates, peers, and other teammates to test out \textit{Amiga}. These audiences ranged in age from 18 to 49, and included a span of demographics. Users liked \textit{Amiga}'s clear and simple design, large text, ``calming'' colors, and user-friendly interface. Some of the feedback received was addressed live, during project execution in other sprints, or remain as to-do tasks for implementation as time permits for the team members.

\section*{Conclusion}

\textit{Amiga} was a project that the team enjoyed working. \textit{Amiga} gave the team a valuable platform upon which to learn React Native, Agile practices, and how to manage a codebase with several contributors. The team also learned about mental health and how it is viewed by a broad spectrum of individuals. In catering to these "business demands," the team also learned how to pivot and adjust existing and future work, all to create a functional deliverable. Team members are proud of this project and intend to showcase \textit{Amiga} on their portfolios.
\newline
\begin{center}
    \textbf{This concludes the technical documentation and report for Project \textit{Amiga}}.
\end{center}

\end{document}